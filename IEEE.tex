\documentclass[conference]{IEEEtran}
\IEEEoverridecommandlockouts
% The preceding line is only needed to identify funding in the first footnote. If that is unneeded, please comment it out.
\usepackage{cite}
\usepackage{amsmath,amssymb,amsfonts}
\usepackage{algorithmic}
\usepackage{graphicx}
\usepackage{textcomp}
\usepackage{xcolor}
\usepackage{hyperref}
\usepackage{nohyperref}
\usepackage{amstex}
\usepackage{enumitem}
\newcommand{\BibTeX}{\textrm{B \kern -.05em \textsc{i \kern -.025em b} \kern -.08em
T \kern -.1667em \lower .7ex \hbox{E} \kern -.125emX}}
\begin{document}

    \title{Knowledge Discovery and Data Mining \hspace{2cm} Finals Report}

    \author{
        \IEEEauthorblockN{1\textsuperscript{st} 522H0036 - Luong Canh Phong}
        \IEEEauthorblockA{
            \textit{Faculty of Information Technology} \\
            \textit{Ton Duc Thang University}\\
            Ho Chi Minh City, Vietnam \\
            522H0036@student.tdtu.edu.vn
        }
        \and
        \IEEEauthorblockN{2\textsuperscript{nd} 520H0341 - Nguyen Thai Bao}
        \IEEEauthorblockA{
            \textit{Faculty of Information Technology} \\
            \textit{Ton Duc Thang University}\\ \
            Ho Chi Minh City, Vietnam \\
            520H0341@student.tdtu.edu.vn
        }
        \and
        \IEEEauthorblockN{3\textsuperscript{rd} 522H0030 - Le Tan Huy}
        \IEEEauthorblockA{
            \textit{Faculty of Information Technology} \\
            \textit{Ton Duc Thang University}\\
            Ho Chi Minh City, Vietnam \\
            522H0030@student.tdtu.edu.vn
        }
        \and
        \IEEEauthorblockN{4\textsuperscript{th} 522H0008 - Dao Minh Phuc}
        \IEEEauthorblockA{
            \textit{Faculty of Information Technology} \\
            \textit{Ton Duc Thang University}\\
            Ho Chi Minh City, Vietnam \\
            522H0008@student.tdtu.edu.vn
        }
        \and
        \IEEEauthorblockN{5\textsuperscript{th} 522H0136 - Nguyen Nhat Phuong Anh}
        \IEEEauthorblockA{
            \textit{Faculty of Information Technology} \\
            \textit{Ton Duc Thang University}\\
            Ho Chi Minh City, Vietnam \\
            5220136@student.tdtu.edu.vn
        }
    }

    \maketitle

    \begin{abstract}
        Spam emails are a common problem seen on the internet as it is an annoyance in daily life and a cyber security risks to any sensitive and important data of a person or an organization/business.
        With the number of spam emails increasing more and more significantly over the past few years, many more algorithms are created and improved in spam detection efficiency.
        Overall, this paper goes through the basic understanding of spam emails, understanding the necessity of a spam classification algorithm, and learn more about the methodologies, its effectiveness and usefulness when detecting spam emails.
    \end{abstract}

    \section{Introduction}
    \label{sec:introduction}
    Since the birth of the Internet, spam email has been a common occurrence.
    Along with the rapid growth and widespread of the Internet, the frequency has been increasing significantly, especially over the past decade.
    In addition to being nuisances, a waste of time and email storage, spam emails can be sent with malicious intent of stealing information, hijacking devices by storing malware within the content of the email itself.
    And with the nature of email spam being sent by botnets, it isn't easy to avoid the situation due to a new bot can be easily created in case another one got blocked or banned on the site.
    A common way how most platforms (such as Gmail, Yahoo!, Outlook) handle these spams is to develop a Machine Learning (ML) model to detect and get rid of the spam emails, lowering the number of spams getting into the inbox.

    \section{Importance of Spam Classification}
    \label{sec:importance-of-spam-classification}
    To understand why spam classification is important to our lives, we must first understand the spam emails and its impact on daily life and businesses.
\subsection{Different Types of Spam Emails}
\label{subsec:different-types-of-spam-emails}

There are various forms of spam, sent with different intentions and purposes.
But they're commonly grouped into:

\begin{itemize}
    \item Phishing Emails: (TBA)
    \item Email Spoofing: (TBA)
    \item Tech support scams: (TBA)
    \item Current event scams: (TBA)
    \item Marketing/advertising email: (TBA)
    \item Malware scam: (TBA)
\end{itemize}

\subsection{Problems with Spam Emails}
\label{subsec:problems-with-spam-emails}

According to statistics report in 2023, 160 billion spam emails are sent every day, which is 46\% of the 347 billion emails sent on a daily basis.
Out of which, the most common type being marketing/advertising emails, which take up around 36\%, followed up with promotional of adult content, around 31.7\% of total spam emails. Despite scam and fraudulent emails is the least common type, over 70\% of them are phishing emails which is still over 6 billion phishing emails are being sent to users daily.

A single spam email carbon emission is almost 0.03g of CO2e, with the amount of spam being sent daily, it can easily get nearly 5 tonnes of CO2e being released every day.
Additionally, two-thirds of spam receivers have been reported to have their mental health affected due to the number of spam or phishing scams.

For businesses, this spam can be sent as a way to get businesses to invest in nonexistent organizations under the disguise of an investment and promised the payback would be worth the money spent, for individuals it would be under the form of bitcoin investment or for a chartiable cause.
Once the money is received, the sender would delete all traces and block the recipient contact.

\section{Methodologies for Spam Classification}
\label{sec:methodologies-for-spam-classification}

Many methods to prevent spam are applied; a commonly used method is using Machine Learning models like Random Forest (RF), Support Vector Machine (SVM), Logistic Regression (LR), Naive Bayes (NB) or Deep Learning models such as Artificial Neural Network (ANN),
(Explantion of used algorithms)

    \section{Visualizing the Data}
    \label{sec:visualizing-the-data}
    \input{sections/visualizing-the-data}

    \section{Preprocessing the data}
    \label{sec:preprocessing-the-data}
    This section focuses on preparing the dataset for machine learning by cleaning and transforming the text data.
It includes removing unnecessary stop words, tokenizing the emails, and vectorizing the emails.

\subsection{Removing unnecessary stopwords}
\label{subsec:removing-unnecessary-stopwords}

\textbf{stop\_words = set(stopwords.words('english'))}: This line creates a set called stop\_words.
It uses the nltk library's stopwords module to get a list of common English words like ``the,'' ``a,'' ``is,'' etc., which are often removed from text data because they usually carry little meaningful information.
These words are converted into a set for efficient lookup during the removal process.

\textbf{puncture = set(string.punctuation)}: This line creates a set called puncture.
It uses the string library's punctuation attribute to get a list of punctuation marks.
Similar to stop\_words, these punctuation marks are converted into a set for efficient removal.

\textbf{def remove\_stopwords(text)}: This line defines a function named remove\_stopwords that takes a single argument, text, representing the input text to be processed.

\textbf{return ` '.join([word for word in text.split() if word not in stop\_words and word not in puncture])}: This line is the core of the function.
It uses a list comprehension to filter out stopwords and punctuation from the input text:

\begin{itemize}
    \item text.split(): This splits the input text into a list of individual words.
    \item if word not in stop\_words and word not in puncture: This condition checks if a word is not in either the stop\_words set or the puncture set.
    \item \text{[}word for \dots if \dots\text{]}: This creates a new list containing only the words that passed the condition.
    \item ` '.join(\ldots): This joins the words back together into a single string, separated by spaces.
\end{itemize}

Finally, the function returns this processed string (with stopwords and punctuation removed).
This step helps to reduce noise and focus on more meaningful words for further analysis.

\subsection{Tokenizing the emails}
\label{subsec:tokenizing-the-emails}

Each email was then tokenized, which involves splitting the text into individual words or tokens.

\textbf{import spacy}: This line imports the spacy library, a popular Python library for NLP tasks.

\textbf{nlp = spacy.load(`en\_core\_web\_sm')}: This line loads a specific language model within spaCy called en\_core\_web\_sm.
This model is trained on a large English text dataset and is used for various NLP operations, including schematization

\subsection{Defining and Applying the Lemmatization Function}
\label{subsec:defining-and-applying-the-lemmatization-function}

The next step involves reducing words to their base or root form, known as lemmatization, using the spaCy library.
This helps standardize words for better analysis.

\textbf{def lemmatize(email):}: This line defines a function named lemmatize that takes a single argument, ``email'', representing the input email text to be processed.

\textbf{doc = nlp(email)}: Inside the function, this line uses the previously loaded spaCy language model (``nlp'') to process the input ``email'' text.
The result, a spaCy Doc object containing tokens and their linguistic features, is stored in the ``doc'' variable.

\textbf{return ` '.join(token.lemma\_ for token in doc)}: This line performs the lemmatization and returns the result.
\begin{itemize}
    \item ``token.lemma\_ for token in doc'': This iterates through each token (word) in the processed ``doc'' object and extracts its lemma (base form).
    For example, ``running'' becomes ``run,'' ``studies'' become ``study.''
    \item `` ` '.join(\ldots)'': This joins the extracted lemmas back together into a single string, separated by spaces.
\end{itemize}
The function returns this lemmatized string.

\textbf{df['email'] = df['email'].apply(lemmatize)}: This line applies the ``lemmatize'' function to each entry in the ``email'' column of the pandas DataFrame named ``df''.
The original email text in the column is replaced by its lemmatized version.

This step is crucial for further analysis, as it allows the model to treat different forms of the same word (e.g., ``run,'' ``running,'' ``ran'') as a single concept, improving pattern recognition.

\subsection{Removing duplicate words}
\label{subsec:removing-duplicate-words}

This code snippet aims to remove duplicate words from each email string within the ``email'' column of the DataFrame ``df'', while importantly maintaining the original order of the remaining words.

\textbf{df[`email'] = \dots}: This part indicates that the result of the operation described on the right side will be assigned back to the ``email'' column of the ``df'' DataFrame, overwriting the previous content.

\textbf{df[`email'].apply(\dots)}: The ``apply()'' method is used here to apply a specific function to each element (each email string) in the ``email'' column.

\textbf{lambda x: \dots}: This defines an anonymous inline function (a lambda function).
The function takes one argument, ``x'', which represents an individual email string from the column during the ``apply'' operation.

\textbf{x.split()}: This part of the lambda function splits the input email string ``x'' into a list of individual words, using spaces as the default delimiter.

\textbf{set(x.split())}: This converts the list of words obtained from ``x.split()'' into a Python set.
A key property of sets is that they only store unique elements, so this step automatically removes any duplicate words present in the list.

\textbf{sorted(\dots, key=x.split().index)}: This is the core part for preserving order.
\begin{itemize}
    \item ``sorted(\ldots)'': This function sorts the elements of the set (which contains only unique words).
    \item ``key=x.split().index'': This is the crucial argument.
    It tells ``sorted'' not to sort alphabetically, but based on the index (position) of each unique word in the *original* list created by ``x.split()''.
    This effectively restores the initial sequence of words.
\end{itemize}

\textbf{` '.join(\dots)}: Finally, this joins the sorted (by original position), unique words back into a single string, with words separated by spaces.

Removing duplicate words can be beneficial in text processing tasks as it helps reduce redundancy and noise in the data, potentially leading to improved model performance.
Maintaining the original word order is important for preserving the contextual meaning of the email.

\subsection{Vectorizing}
\label{subsec:vectorizing}

To represent the processed email text numerically, a technique called vectorization was employed.
Specifically, TF-IDF (Term Frequency-Inverse Document Frequency) vectorization was used to convert the text into a numerical matrix that machine learning models can understand.
This step is crucial for training models to recognize patterns and make predictions based on email content.

\textbf{Creating the Vectorizer}

\textbf{TfidfVectorizer}: This line imports or initializes the ``TfidfVectorizer'' class from the ``sklearn.feature\_extraction.text'' library.
This class converts a collection of raw documents to a matrix of TF-IDF features.

\textbf{min\_df=1}: This parameter is set during the initialization of ``TfidfVectorizer''.
It specifies that a word must appear in at least one document (email, in this case) to be considered part of the vocabulary.
Setting it to 1 includes all words that appear at least once (unless filtered by other parameters like ``max\_features'').

\textbf{max\_features=1000}: This parameter limits the size of the vocabulary to the 1000 most frequent words across all emails, based on term frequency.
This helps manage the dimensionality of the resulting vector space and focuses on potentially more informative terms.

\textbf{Fitting and Transforming the Data:}

\textbf{fit\_transform(df['email'])}: This method of the ``TfidfVectorizer'' object is called on the ``email'' column of the DataFrame.
It performs two actions in sequence:
\begin{itemize}
    \item \textbf{fit}: It learns the vocabulary (the 1000 most frequent words meeting the ``min\_df'' criteria) from the provided email data.
    It also calculates the Inverse Document Frequency (IDF) for each word in the learned vocabulary.
    \item \textbf{transform}: It converts each email in the ``df['email']'' column into a numerical vector.
    Each element in the vector corresponds to a word in the vocabulary, and its value is the TF-IDF score for that word in that specific email.
    The result is typically a sparse matrix (``vector\_matrix'' in the context description).
\end{itemize}

\textbf{Converting to DataFrame:}

\textbf{vector\_df = pd.DataFrame(vector\_matrix.toarray(), columns=vectorizer.get\_feature\_names\_out())}: This line converts the numerical representation into a more readable pandas DataFrame.
\begin{itemize}
    \item ``vector\_matrix.toarray()'': The TF-IDF process often returns a sparse matrix (efficient for storing data with many zeros).
    This method converts it into a standard dense NumPy array.
    \item ``pd.DataFrame(\ldots)'': This creates a pandas DataFrame from the dense array.
    \item ``columns=vectorizer.get\_feature\_names\_out()'': This sets the column names of the new DataFrame (``vector\_df'') to be the actual words (features) that the vectorizer learned during the ``fit'' step.
\end{itemize}
This DataFrame (``vector\_df'') now holds the vectorized data, where each row represents an email, each column represents a unique word (feature), and the cell values are the TF-IDF scores.

\textbf{Previewing the Result:}

\textbf{vector\_df.head()}: This line calls the ``head()'' method on the newly created ``vector\_df'' DataFrame.
It displays the first few rows (typically 5) of the DataFrame, allowing for a quick inspection of the vectorized data structure and the TF-IDF values.

The values in the cells represent the TF-IDF score, indicating the calculated importance of each word (column) in each email (row), considering both its frequency in that email and its rarity across all emails.
This numerical representation is the final input format used for training machine learning models for tasks like spam detection.

    \section{Models in use}
    \label{sec:models-in-use}
    The project employs six different models for email classification, including five traditional machine learning models and one deep learning model.
Below is a detailed explanation of each model, its algorithm, implementation, and results.

\subsection{Random Forest Model}
\label{subsec:random-forest-model}
Random Forest is a popular machine learning model in the Ensemble Learning family, used for both classification and regression tasks.
It consists of a collection of decision trees trained on different subsets of data, with the final prediction aggregated from these trees.
The core idea is to combine multiple weak learners (individual decision trees) into a strong learner with higher accuracy and reduced overfitting compared to a single decision tree.

Random Forest relies on two key principles:

\begin{itemize}
    \item \textbf{Bagging (Bootstrap Aggregating)}: Creates multiple data subsets by randomly sampling with replacement from the original dataset.
    \item \textbf{Feature Randomness}: At each split in a decision tree, only a random subset of features is considered.
\end{itemize}

\subsubsection{Basic Components}\text{}

\smallskip
\textbf{Decision Tree}: The fundamental unit of Random Forest.
Each tree is built independently on a data subset.

\textbf{Forest}: A collection of many decision trees (typically hundreds or thousands).

\textbf{Bootstrap Sample}: Each tree is trained on a randomly sampled subset of the original dataset.

\textbf{Random Feature Subset}: At each node, only a random subset of features is evaluated for the best split.

\smallskip

\subsubsection{Detailed Algorithm}\text{}

\begin{enumerate}[label=Step \arabic*:, align=left, leftmargin=20pt,labelsep=1em]
    \item Data Preparation
    \begin{itemize}
        \item Assume an original dataset $D$ with $N$ samples and $M$ features.
    \end{itemize}

    \item Create Bootstrap Samples
    \begin{itemize}
        \item Generate $T$ subsets $D1, D2, \ldots,DT$ (where $T$ is the number of trees), each of size $N$, by randomly sampling with replacement from $D$.
        \item Due to sampling with replacement, some samples may appear multiple times, while others may not appear (approximately 36.8\% of the data is left out, called Out-of-Bag (OOB) data).
    \end{itemize}

    \item Build Each Decision Tree
    \begin{itemize}
        \item For each subset $D_t$:
        \begin{enumerate}
            \item Initialize a decision tree: Trees are grown fully without pruning.
            \item At each node:
            \begin{itemize}
                \item Randomly select a subset of $m$ features from the total $M$ features ($m < M$). Typically:
                \begin{itemize}
                    \item For classification: $m = \sqrt{M}$
                    \item For regression: $m = M/3$
                \end{itemize}
                \item Find the best feature among these $m$ features to split the node, based on criteria like Gini Index, Entropy (classification), or Mean Squared Error (regression).
            \end{itemize}
            \item Repeat the splitting process until the tree is complete (reaches maximum depth or cannot split further).
        \end{enumerate}
        \item Result: $T$ independent decision trees.
    \end{itemize}

    \item Aggregate Results

    \smallskip
    For classification:
    \begin{itemize}
        \item Each tree predicts a class.
        \item The final class is the one with the most votes (Majority Voting).
    \end{itemize}

    \smallskip
    For regression:
    \begin{itemize}
        \item Each tree predicts a numerical value.
        \item The final value is the average of all predictions.
    \end{itemize}

    \item Model Evaluation (Using OOB Error)
    \begin{itemize}
        \item OOB data (not used to train a given tree) is used to test the performance of each tree.
        \item Aggregate OOB results to estimate the overall accuracy of the Random Forest without a separate test set.
    \end{itemize}
\end{enumerate}

\smallskip

\subsubsection{Mathematical Formulas}\text{}

There are splitting criteria at each node in the tree, and it is different for each task.

\textbf{For Classification:}
\begin{itemize}
    \item Gini Index:
    \[Gini = 1 - \sum_{i=1}^C{p_i^2}\]
    where $p_i$ is the proportion of class $i$ in the node.
    \item Entropy:
    \[Entropy = - \sum_{i=1}^C{p_i \log_2{p_i}}\]
    \item Final Result:
    \[\hat{y} = mode(\hat{y}_1, \hat{y}_2, \ldots, \hat{y}_t)\]
    where $\hat{y}_t$ is the prediction from a tree $t$.
\end{itemize}

\textbf{For Regression:}
\begin{itemize}
    \item Splitting criterion: Minimize Mean Squared Error:
    \[MSE = \frac{1}{n} \sum_{i=1}^n{(y_i - \hat{y}_i)^2}\]
    \item Final Result:
    \[\hat{y} = \frac{1}{T} \sum_{t=1}^T{\hat{y}_t}\]
\end{itemize}

\smallskip

\subsubsection{Advantages}

\begin{itemize}
    \item High accuracy due to combining multiple trees.
    \item Resistance to overfitting thanks to randomness and aggregation.
    \item Handles large datasets with many features and samples effectively.
    \item No need for data normalization since it is based on decision trees.
    \item Provides feature importance measurement based on impurity reduction.
\end{itemize}

\smallskip

\subsubsection{Disadvantages}

\begin{itemize}
    \item Resource-intensive: Requires significant memory and computation for large numbers of trees.
    \item Less interpretable than a single decision tree.
    \item Reduced performance with linear data, where linear regression might be more suitable.
\end{itemize}

\subsection{Support Vector Machine (SVM)}
\label{subsec:support-vector-machine}
SVM is a supervised machine learning algorithm used for classification and regression, though it is primarily applied to classification tasks.
The core idea of SVM is to find the optimal hyperplane that best separates data points of different classes in a high-dimensional space, maximizing the margin between the classes.
For cases where data is not linearly separable, SVM uses the kernel trick to transform the data into a higher-dimensional space where a linear boundary can be established.

SVM is known for its robustness and effectiveness in handling high-dimensional datasets and is widely used in tasks like text classification, image recognition, and bioinformatics.

\subsubsection{Basic Components}\textbf{}

\smallskip
\textbf{Hyperplane}: A decision boundary that separates data points of different classes.
In d-dimensional space, a hyperplane is defined by $w^t +b = 0$, where w is the weight vector, b is the bias, and x is the input vector.

\textbf{Margin}: The distance between the hyperplane and the nearest data point from either class.
SVM aims to maximize this margin.

\textbf{Support Vectors}: The data points closest to the hyperplane, which define the margin and are critical to determining the hyperplane’s position.

\textbf{Kernel Function}: A function that transforms non-linearly separable data into a higher-dimensional space where a linear boundary can be found (e.g., linear, polynomial, or radial basis function (RBF) kernels).

\textbf{Regularization Parameter $(C)$}: Controls the trade-off between maximizing the margin and minimizing classification errors.

\textbf{Slack Variables}: Allow for some misclassification in soft-margin SVM to handle non-linearly separable data.

\smallskip

\subsubsection{Detailed Algorithm}\textbf{}

\begin{enumerate}[label=Step \arabic*:, align=left, leftmargin=20pt,labelsep=1em]
    \item Data Preparation
    \begin{itemize}
        \item Assume an original dataset
        \[D = \{ (x_1, y_1), (x_2, y_2), \dots, (x_N, y_N) \}\]
        where $x_i \in \mathbb{R}^d$ (feature vector with $d$ dimensions) and $y_i \in \{-1, +1\}$ (binary class labels).
    \end{itemize}

    \item Define the Objective

    \smallskip
    Hard-Margin SVM (Linearly Separable Data):
    \begin{itemize}
        \item Find the hyperplane $w^T x + b = 0$ that separates the classes with the maximum margin.
        \item The margin is defined as the distance from the hyperplane to the nearest data point, given by $\frac{2}{\|w\|}$, where $\|w\| = \sqrt{w^T w}$.
        \item Objective: Maximize the margin, i.e., minimize $\frac{1}{2} \|w\|^2$, subject to:
        \[y_i (w^T x_i + b) \geq 1, \quad \forall i = 1, \dots, N\]
    \end{itemize}

    \smallskip
    Soft-Margin SVM (Non-Linearly Separable Data):
    \begin{itemize}
        \item Introduce slack variables $\xi_i \geq 0$ to allow some misclassification.
        \item Objective: Minimize:
        \[\frac{1}{2} \|w\|^2 + C \sum_{i=1}^N \xi_i\]
        Subject to:
        \[y_i (w^T x_i + b) \geq 1 - \xi_i, \quad \xi_i \geq 0, \quad \forall i\]
        where $C$ is the regularization parameter controlling the trade-off between margin maximization and classification error.
    \end{itemize}

    \item Solve the Optimization Problem

    \smallskip
    The optimization problem is typically solved in its \textbf{dual form} using Lagrange multipliers to handle constraints efficiently.
    \begin{itemize}
        \item Dual Problem:
        \begin{itemize}
            \item Introduce Lagrange multipliers $\alpha_i \geq 0$.
            \item The dual optimization problem is:
        \end{itemize}

        Maximize
        \[L(\alpha) = \sum_{i=1}^N \alpha_i - \frac{1}{2} \sum_{i=1}^N \sum_{j=1}^N \alpha_i \alpha_j y_i y_j (x_i^T x_j)\]

        Subject to:
        \[\sum_{i=1}^N \alpha_i y_i = 0, \quad 0 \leq \alpha_i \leq C, \quad \forall i\]

        \item The solution $\alpha_i$ determines the weight vector:
        \[w = \sum_{i=1}^N \alpha_i y_i x_i\]
        \item The bias $b$ is computed using support vectors (where $\alpha_i > 0$).
        \item Support vectors are the points where $y_i(w^T x_i + b) = 1$ (on the margin) or $\xi_i > 0$ (misclassified or within the margin).
    \end{itemize}

    \item Handle Non-Linear Data with Kernels
    For non-linearly separable data, map the input data to a higher-dimensional space using a kernel function $K(x_i, x_j)$.
    \begin{itemize}
        \item Common kernels:
        \begin{itemize}
            \item Linear Kernel:
            \[K(x_i, x_j) = x_i^T x_j\]
            \item Polynomial Kernel:
            \[K(x_i, x_j) = (x_i^T x_j + c)^d\]
            \item Radial Basis Function (RBF) Kernel:
            \[K(x_i, x_j) = \exp(-\gamma \|x_i - x_j\|^2)\]
        \end{itemize}

        \item The dual problem becomes: Maximize
        \[L(\alpha) = \sum_{i=1}^N \alpha_i - \frac{1}{2} \sum_{i=1}^N \sum_{j=1}^N \alpha_i \alpha_j y_i y_j K(x_i, x_j)\]
        Subject to the same constraints as above.
    \end{itemize}

    \item Prediction

    For a new input $x$:
    \begin{itemize}
        \item Compute the decision function:
        \[f(x) = \sum_{i \in \text{SV}} \alpha_i y_i K(x_i, x) + b\]
        where SV is the set of support vectors.
        \smallskip
        \item Predict the class:
        \[\hat{y} = sign(f(x))\]
        \item For probability estimates (if enabled), use techniques like Platt scaling to convert f(x) into probabilities
    \end{itemize}
\end{enumerate}

\smallskip

\subsubsection{Advantages}\textbf{}

\begin{itemize}
    \item Effective in High-Dimensional Spaces: Works well with datasets having many features (e.g., text classification).
    \item Robust to Outliers: Maximizing the margin focuses on support vectors, ignoring points far from the boundary.
    \item Flexible with Kernels: The kernel trick allows SVM to handle non-linearly separable data effectively.
    \item Global Optimization: The convex optimization problem ensures a unique solution.
    \item Sparse Solution: Only support vectors (a subset of the data) determine the model, making it memory-efficient.
\end{itemize}

\subsubsection{Disadvantages}\textbf{}

\begin{itemize}
    \item Computationally Expensive: Training time scales poorly with large datasets ($O(N^2) to O(N^3)$ for solving the quadratic optimization problem).
    \item Sensitive to Parameter Tuning: Requires careful tuning of $C$ and kernel parameters (e.g., $\gamma$ for RBF kernel).
    \item Not Interpretable: The resulting hyperplane and support vectors are not as intuitive as decision trees.
    \item Poor with Noisy Data: Overlapping classes or noisy data can degrade performance.
    \item Not Ideal for Large Datasets: Due to high computational cost, SVM is less suitable for massive datasets compared to models like Random Fores
\end{itemize}

\subsection{K-Nearest Neighbors (KNN)}
\label{subsec:k-nearest-neighbors}
KNN is a simple, non-parametric, and instance-based supervised machine learning algorithm used for both classification and regression.
It works by finding the K closest data points (neighbors) to a new input in the feature space and making a prediction based on their labels (for classification) or values (for regression).
KNN is often described as a ``lazy learning'' algorithm because it does not build an explicit model during training; instead, it stores the training data and performs calculations at prediction time.

The core idea is: ``A point is likely to belong to the same class (or have a similar value) as its nearest neighbors.''

\subsubsection{Basic Components}\text{}

\smallskip
\textbf{Training Data}: The entire dataset
    \[D = {(x_1, y_1), (x_2, y_2), \ldots, (x_N, y_N)}\]
    where $x_i \in \mathbb{R}^d$ is a feature vector with $d$ dimensions, and $y_i$ is the label (for classification, e.g., $y_i \in \{0,1\}$) or value (for regression, e.g., $y_i \in \mathbb{R}$)

\textbf{Distance Metric}: A function to measure the ``closeness'' between points, typically Euclidean distance.

\smallskip
\subsubsection{Detailed Algorithm}\text{}

Here are the steps to implement and use the KNN algorithm:

\begin{enumerate}[label=Step \arabic*:, align=left, leftmargin=20pt,labelsep=1em]
    \item Data Preparation
    \begin{itemize}
        \item Prepare the dataset
            \[D = \{(x_1, y_1), (x_2, y_2), \dots, (x_N, y_N)\}\]
            where $x_i$ are feature vectors and $y_i$ are labels (classification) or continuous values (regression).
        \item Normalize or standardize features, as KNN relies on distance calculations, and differing feature scales can skew results.
    \end{itemize}

    \item Choose Parameters
    \begin{itemize}
        \item Select $K$, the number of neighbors to consider.
        \item Choose a \textbf{distance metric} (e.g., Euclidean, Manhattan).
        \item \textbf{For classification}, decide on a voting method (e.g., majority voting or weighted voting).
        \item \textbf{For regression}, decide on an aggregation method (e.g., mean or weighted mean).
    \end{itemize}

    \item Prediction
    \begin{itemize}
        \item For a new input $x$:
        \begin{enumerate}
            \item Compute the distance between $x$ and all points $x_i$ in the training set using the chosen distance metric.
            \item Identify the K nearest neighbors (the $K$ points with the smallest distances).
            \item \textbf{Classification}: Predict the class by majority voting among the $K$ neighbors' labels.
            \item \textbf{Regression}: Predict the value by averaging the $K$ neighbors' values (or using a weighted average).
        \end{enumerate}
        \item Weighted KNN: Assign weights to neighbors based on their distance (e.g., inverse distance), giving closer neighbors more influence.
    \end{itemize}
\end{enumerate}

\subsubsection{Mathematical Formulas}\text{}

\smallskip
KNN's mathematics is centered around \textbf{distance calculations} and \textbf{aggregation} of neighbors' outputs.
Below are the key formulas:

\begin{itemize}
    \item Euclidean Distance (L2 Norm):
        \[d(x, x_i) = \sqrt{\sum_{j=1}^d (x_j - x_{i,j})^2}\]
        where $x_j$ and $x_{i,j}$ are the $j$-th features of points $x$ and $x_i$, and $d$ is the number of features.

    \item Manhattan Distance (L1 Norm):
        \[d(x, x_i) = \sum_{j=1}^d |x_j - x_{i,j}|\]

    \item Minkowski Distance (Generalization):
        \[d(x, x_i) = \left( \sum_{j=1}^d |x_j - x_{i,j}|^p \right)^{1/p}\]

    \begin{itemize}
        \item $p = 2$: Euclidean distance.
        \item $p = 1$: Manhattan distance.
    \end{itemize}

    \item Cosine Similarity (for high-dimensional data):
        \[d(x, x_i) = 1 - \text{cosine similarity} = 1 - \frac{x \cdot x_i}{\|x\| \|x_i\|}\]

    \item Weighted Distance (if features have different importance):
        \[d(x, x_i) = \sqrt{\sum_{j=1}^d w_j (x_j - x_{i,j})^2}\]
        where $w_j$ is the weight for feature $j$.
\end{itemize}

\smallskip
\textbf{Classification Prediction}

\begin{itemize}
    \item Majority Voting:
        \[\hat{y} = \text{mode}(y_{i1}, y_{i2}, \dots, y_{iK})\]
        where $y_{i1}, \dots, y_{iK}$ are the labels of the $K$ nearest neighbors.

    \item Weighted Voting:
        \[\hat{y} = \arg\max_c \sum_{i \in \text{KNN}} w_i I(y_i = c)\]
        where:
    \begin{itemize}
        \item $w_i = \frac{1}{d(x, x_i)}$ (inverse distance) or another weighting function.
        \item $I(y_i = c) = 1$ if $y_i = c$, else 0.
        \item $c$ is a class label.
    \end{itemize}
\end{itemize}

\smallskip
\textbf{Regression Prediction}

\begin{itemize}
    \item Mean:
        \[\hat{y} = \frac{1}{K} \sum_{i \in \text{KNN}} y_i\]

    \item Weighted Mean:
        \[\hat{y} = \frac{\sum_{i \in \text{KNN}} w_i y_i}{\sum_{i \in \text{KNN}} w_i}\]
        where $w_i = \frac{1}{d(x, x_i)}$ or similar.
\end{itemize}

\smallskip
\textbf{Error Metrics}
\begin{itemize}
    \item Classification Error:
        \[\text{Error} = \frac{1}{N_{\text{test}}} \sum_{i=1}^{N_{\text{test}}} I(\hat{y}_i \neq y_i)\]

    \item Regression Error (Mean Squared Error):
        \[\text{MSE} = \frac{1}{N_{\text{test}}} \sum_{i=1}^{N_{\text{test}}} (\hat{y}_i - y_i)^2\]
\end{itemize}

\subsubsection{Advantages}

\begin{itemize}
    \item Simple and Intuitive: Easy to understand and implement.
    \item No Training Phase: No model is built, making it flexible for dynamic datasets.
    \item Non-Parametric: Makes no assumptions about data distribution, effective for non-linear patterns.
    \item Versatile: Works for both classification and regression.
    \item Robust to Multimodal Data: Can handle complex decision boundaries.
\end{itemize}

\subsubsection{Disadvantages}
\begin{itemize}
    \item Computationally Expensive at Prediction Time: Requires calculating distances to all training points ($O(Nd)$ per prediction, where $N$ is the number of samples, $d$ is the number of features).
    \item Memory-Intensive: Stores the entire training dataset.
    \item Sensitive to Noise and Outliers: Noisy points can skew predictions.
    \item Curse of Dimensionality: Performance degrades in high-dimensional spaces due to sparse data.
    \item Requires Feature Scaling: Distances are sensitive to feature scales.
\end{itemize}

\subsection{Naive Bayes}
\label{subsec:naive-bayes}
Naive Bayes is a family of simple, probabilistic, supervised machine learning algorithms used primarily for classification tasks, though it can be adapted for other purposes.
It is based on Bayes’ Theorem and assumes that features are conditionally independent given the class label (the ``naive'' assumption).
Despite this simplifying assumption, Naive Bayes performs surprisingly well in many real-world applications, especially in text classification and spam filtering.

The core idea is: Given a set of features, predict the most likely class by calculating probabilities based on prior observations.

\subsubsection{Basic Components}

\begin{itemize}
    \item \textbf{Training Data}:
        \[D = \{(x_1, y_1), (x_2, y_2), \dots, (x_N, y_N)\}\]
        where $x_i = (x_{i1}, x_{i2}, \dots, x_{id}) \in \mathbb{R}^d$ is a feature vector with d dimensions, and $y_i \in \{C_1, C_2, \dots, C_k\}$ is a class label from k classes.
    \item \textbf{Bayes' Theorem}: The foundation for calculating probabilities of classes given features.
    \item \textbf{Conditional Independence Assumption}: Assumes that features $x_{i1}, x_{i2}, \dots, x_{id}$ are independent given the class $y$.
    \item \textbf{Prior Probability}: The probability of each class before observing the features.
    \item \textbf{Likelihood}: The probability of observing the features given a class.
    \item \textbf{Posterior Probability}: The probability of a class given the observed features, which is used for prediction.
    \item \textbf{Smoothing}: A technique (e.g., Laplace smoothing) to handle zero probabilities for unseen feature-class combinations.
\end{itemize}

\subsubsection{Detailed Algorithm}

Here are the steps to implement and use the Naive Bayes algorithm:

\begin{enumerate}[label=Step \arabic*:, align=left, leftmargin=20pt,labelsep=1em]
    \item Data Preparation
    \begin{itemize}
        \item Prepare the dataset $D$, where each sample has features $x_i$ and a class label $y_i$.
        \item For categorical features, Naive Bayes is straightforward.
        For continuous features, assume a distribution (e.g., Gaussian) or discretize the data.
    \end{itemize}

    \item Training (Model Building)
    \begin{itemize}
        \item Estimate Prior Probabilities: Calculate the probability of each class $C_j$:
        \[P(C_j) = \frac{\text{Number of samples with class } C_j}{\text{Total number of samples}}\]

        \item Estimate Likelihoods: For each feature $x_m$ and class $C_j$, compute the conditional probability $P(x_m | C_j)$:
        \begin{itemize}
            \item Categorical Features: Use frequency counts.
            \item Continuous Features: Assume a distribution (e.g., Gaussian) and estimate parameters (mean, variance).
            \item Apply smoothing (e.g., Laplace smoothing) to avoid zero probabilities.
        \end{itemize}

        \item Store these probabilities for use during prediction.
    \end{itemize}

    \item Prediction
    \begin{itemize}
        \item For a new input $x = (x_1, x_2, \dots, x_d)$:
        \begin{enumerate}
            \item Compute the posterior probability for each class $C_j$ using Bayes' Theorem:
                \[P(C_j | x) \propto P(C_j) \prod_{m=1}^d P(x_m | C_j)\]
                (The proportionality $\propto$ is used because the denominator P(x) is constant across classes.)

            \item Predict the class with the highest posterior probability:
                \[\hat{y} = \arg\max_{C_j} P(C_j) \prod_{m=1}^d P(x_m | C_j)\]

            \item Optionally, compute normalized probabilities by dividing by the sum of all class posteriors
        \end{enumerate}
    \end{itemize}
\end{enumerate}


\subsubsection{Mathematical Formulas}\text{}

\smallskip
Naive Bayes is grounded in Bayes' Theorem and the conditional independence assumption.
Below are the key formulas, explained intuitively.

\textbf{Bayes' Theorem}

For a class $C_j$ and feature vector $x = (x_1, x_2, \dots, x_d)$:
\[P(C_j | x) = \frac{P(C_j) P(x | C_j)}{P(x)}\]

\begin{itemize}[leftmargin=*] % Start list explaining terms
    \item $P(C_j | x)$: Posterior probability (probability of class $C_j$ given features $x$).
    \item $P(C_j)$: Prior probability (probability of class $C_j$ before seeing $x$).
    \item $P(x | C_j)$: Likelihood (probability of observing $x$ given class $C_j$).
    \item $P(x)$: Evidence (probability of observing $x$, a normalizing constant).
\end{itemize}

Since $P(x)$ is the same for all classes, we can ignore it for classification:
\[P(C_j | x) \propto P(C_j) P(x | C_j)\]

\textbf{Conditional Independence Assumption}

Naive Bayes assumes that features are independent given the class:
\[P(x | C_j) = P(x_1, x_2, \ldots, x_d | C_j) = \prod_{m=1}^d P(x_m | C_j)\]

\textbf{Intuition:} If you know the class (e.g., ``spam email''), the presence of one feature (e.g., the word ``free'') doesn't affect the probability of another feature (e.g., the word ``win'').
This is often unrealistic but simplifies calculations.

\smallskip
\textbf{Prior Probability}
\[P(C_j) = \frac{\text{Count}(y = C_j)}{N}\]
where $\text{Count}(y = C_j)$ is the number of samples with class $C_j$, and $N$ is the total number of samples.

\smallskip
\textbf{Likelihood}

\begin{itemize}
    \item Categorical Features:
        \[P(x_m = v | C_j) = \frac{\text{Count}(x_m = v, y = C_j)}{\text{Count}(y = C_j)}\]
        where $v$ is a specific value of feature $x_m$.

    \item Laplace Smoothing: To avoid zero probabilities:
        \[P(x_m = v | C_j) = \frac{\text{Count}(x_m = v, y = C_j) + \alpha}{\text{Count}(y = C_j) + \alpha \cdot |\text{Values}(x_m)|}\]
        where $\alpha$ (e.g., 1) is the smoothing parameter, and $|\text{Values}(x_m)|$ is the number of possible values for $x_m$.

    \item Continuous Features (Gaussian Naive Bayes): Assume feature $x_m$ follows a Gaussian distribution for class $C_j$:
          \[P(x_m | C_j) = \frac{1}{\sqrt{2\pi \sigma^2_{mj}}} \exp\left( -\frac{(x_m - \mu_{mj})^2}{2\sigma^2_{mj}} \right)\]
          where:
    \begin{itemize}
        \item $\mu_{mj}$: Mean of feature $x_m$ for class $C_j$.
        \item $\sigma^2_{mj}$: Variance of feature $x_m$ for class $C_j$.
    \end{itemize}
\end{itemize}

\smallskip
\textbf{Error Metrics}
\begin{itemize}
    \item Classification Error:
        \[\text{Error} = \frac{1}{N_{\text{test}}} \sum_{i=1}^{N_{\text{test}}} I(\hat{y}_i \neq y_i)\]

    \item Log Loss (if probabilities are used):
        \[\text{Log Loss} = -\frac{1}{N_{\text{test}}} \sum_{i=1}^{N_{\text{test}}} \sum_{j=1}^{k} I(y_i = C_j) \log P(C_j | x_i)\]
\end{itemize}


\subsubsection{Advantages}

\smallskip
\begin{itemize}
    \item Simple and Fast: Easy to implement and computationally efficient, especially for training.
    \item Effective with Small Datasets: Performs well even with limited data, unlike complex models.
    \item Handles High-Dimensional Data: Common in text classification (e.g., bag-of-words models).
    \item Probabilistic Output: Provides probability estimates for each class, useful for decision-making.
    \item Robust to Irrelevant Features: The independence assumption mitigates the impact of irrelevant features.
\end{itemize}

\subsubsection{Disadvantages}
\begin{itemize}
    \item Naive Assumption: The conditional independence assumption is often unrealistic, leading to suboptimal performance when features are correlated.
    \item Sensitive to Zero Probabilities: Requires smoothing to handle unseen feature-class combinations.
    \item Poor with Continuous Features: Gaussian assumptions may not fit all data distributions.
    \item Outperformed by Complex Models: Often less accurate than SVM or Random Forest on complex datasets.
    \item Imbalanced Data Issues: May favor majority classes without proper adjustments.
\end{itemize}


\subsection{XGBoost}
\label{subsec:xgboost}
XGBoost (Extreme Gradient Boosting) is a powerful, scalable, and highly optimized machine learning algorithm used for both classification and regression tasks, though it excels in structured/tabular data.
It belongs to the family of gradient boosting methods, which build an ensemble of decision trees sequentially, where each tree corrects the errors of the previous ones.
XGBoost enhances gradient boosting with advanced regularization, parallel processing, and handling of missing data, making it a go-to choice in data science competitions and real-world applications.

The core idea is: Combine many weak decision trees (weak learners) into a strong predictive model by iteratively minimizing a loss function using gradient-based optimization.

\subsubsection{Basic Components}
\begin{itemize}
    \item \textbf{Decision Trees}: The base learners in XGBoost, typically shallow trees (e.g., depth 3–10) to prevent overfitting.
    \item \textbf{Ensemble}: A collection of trees whose predictions are combined (summed for regression, aggregated for classification).
    \item \textbf{Loss Function}: Measures the difference between predicted and actual values (e.g., mean squared error for regression, log loss for classification).
    \item \textbf{Regularization}: Penalties on tree complexity to prevent overfitting (e.g., L1/L2 regularization on leaf weights).
    \item \textbf{Gradient and Hessian}: First-order (gradient) and second-order (Hessian) derivatives of the loss function guide tree construction.
    \item \textbf{Boosting}: Sequential addition of trees, where each tree focuses on correcting the residuals (errors) of the previous trees.
    \item \textbf{Hyperparameters}: Parameters like learning rate, max depth, and regularization terms control model behavior.
\end{itemize}

\subsubsection{Detailed Algorithm}

\begin{enumerate}[label=Step \arabic*:, align=left, leftmargin=20pt,labelsep=1em]
    \item Data Preparation
    \begin{itemize}
        \item Prepare the dataset $D = \{(x_1, y_1), (x_2, y_2), \dots, (x_N, y_N)\}$, where $x_i \in \mathbb{R}^d$ is a feature vector with d dimensions, and $y_i$ is a label (e.g., $y_i \in \{0, 1\}$ for binary classification, $y_i \in \mathbb{R}$ for regression).
        \item Handle missing values (XGBoost can automatically learn how to treat them).
        \item No feature scaling is required, as XGBoost is tree-based.
    \end{itemize}

    \item Initialize the Model
    \begin{itemize}
        \item Start with an initial prediction (e.g., mean of target values for regression, log-odds for classification).
        \item Define the loss function (e.g., MSE for regression, log loss for classification) and regularization terms.
    \end{itemize}

    \item Build Trees Sequentially
    \begin{itemize}
        \item For T iterations (number of trees):
        \begin{enumerate}
            \item Compute \textbf{gradients} (first derivative of the loss with respect to predictions) and \textbf{Hessians} (second derivative) for each sample.
            \item Build a decision tree to fit the gradients, using a specialized objective function that balances loss reduction and tree complexity.
            \item Update the model's predictions by adding the new tree's output, scaled by a \textbf{learning rate} ($\eta$).
        \end{enumerate}
        \item Each tree focuses on the residuals (errors) of the current model.
    \end{itemize}

    \item Prediction
    \begin{itemize}
        \item For a new input $x$:
        \begin{itemize}
            \item Sum the predictions from all trees (for regression) or compute a weighted sum and apply a sigmoid function (for classification).
            \item Output the final class or value.
        \end{itemize}
    \end{itemize}
\end{enumerate}

\subsubsection{Mathematical Formulas}

XGBoost's mathematics combines \textbf{gradient boosting}, \textbf{decision tree construction}, and \textbf{regularized optimization}.
Below are the key formulas, explained intuitively.

\smallskip
\textbf{Objective Function}

The goal is to minimize a loss function plus regularization:
\[\text{Obj} = \sum_{i=1}^N \ell(y_i, \hat{y}_i) + \sum_{t=1}^T \Omega(f_t)\]

\begin{itemize}
    \item $\ell(y_i, \hat{y}_i)$: Loss function measuring the error between true label $y_i$ and prediction $\hat{y}_i$.
    \begin{itemize}
        \item Regression (MSE):
        \[
            \ell(y_i, \hat{y}_i) = \frac{1}{2}(y_i - \hat{y}_i)^2
        \]
        \item Classification (Log Loss):
        \[
            \ell(y_i, \hat{y}_i) = -[y_i \log(\hat{y}_i) + (1 - y_i) \log(1 - \hat{y}_i)]
        \]
    \end{itemize}

    \item $\Omega(f_t)$: Regularization term for tree $t$:
    \[
        \Omega(f_t) = \gamma T + \frac{1}{2}\lambda \sum_{j=1}^T w_j^2 + \alpha \sum_{j=1}^T |w_j|
    \]
    \begin{itemize}
        \item $T$: Number of leaves in the tree.
        \item $w_j$: Leaf weight (output value of leaf $j$).
        \item $\gamma$: Penalty for adding leaves (controls tree size).
        \item $\lambda$: L2 regularization on leaf weights.
        \item $\alpha$: L1 regularization on leaf weights.
    \end{itemize}
\end{itemize}

\smallskip
\textbf{Gradient Boosting}

Predictions are the sum of outputs from T trees:
\[
    \hat{y}_i = \sum_{t=1}^T f_t(x_i)
\]
where $f_t(x_i)$ is the output of tree $t$ for sample $x_i$.

At iteration t, the prediction is:
\[
    \hat{y}_i^{(t)} = \hat{y}_i^{(t-1)} + \eta f_t(x_i)
\]

\begin{itemize}
    \item $\eta$: Learning rate (shrinks the contribution of each tree to prevent overfitting).
\end{itemize}

\smallskip
\textbf{Gradient and Hessian}

For each sample i, compute:
\begin{itemize}
    \item Gradient (first derivative): $g_i = \frac{\partial \ell(y_i, \hat{y}_i^{(t-1)})}{\partial \hat{y}_i^{(t-1)}}$
    \begin{itemize}
        \item For MSE: $g_i = \hat{y}_i^{(t-1)} - y_i$.
        \item For log loss: $g_i = \hat{y}_i^{(t-1)} - y_i$, where $\hat{y}_i^{(t-1)}$ is the predicted probability.
    \end{itemize}
    \medskip
    \item Hessian (second derivative): $h_i = \frac{\partial^2 \ell(y_i, \hat{y}_i^{(t-1)})}{\partial (\hat{y}_i^{(t-1)})^2}$
    % Nested list for specific cases
    \begin{itemize}
        \item For MSE: $h_i = 1$.
        \item For log loss: $h_i = \hat{y}_i^{(t-1)} (1 - \hat{y}_i^{(t-1)})$.
    \end{itemize}
\end{itemize}

These guide the tree to focus on samples with larger errors.

\textbf{Tree Construction}

Each tree is built to minimize:
\[
    \text{Obj}^{(t)} = \sum_{i=1}^N [g_i f_t(x_i) + \frac{1}{2} h_i f_t(x_i)^2] + \Omega(f_t)
\]

\textbf{Intuition:} The tree predicts values that reduce the loss (via gradients) while keeping the tree simple (via regularization).

\begin{itemize}
    \item For a leaf $j$ with samples $I_j$, the optimal leaf weight is:
    \[
        w_j^* = -\frac{\sum_{i \in I_j} g_i}{\sum_{i \in I_j} h_i + \lambda}
    \]

    \item The gain from splitting a node into left ($I_L$) and right ($I_R$) branches is:
    \[
        \text{Gain} = \frac{1}{2} \left[ \frac{(\sum_{i \in I_L} g_i)^2}{\sum_{i \in I_L} h_i + \lambda} + \frac{(\sum_{i \in I_R} g_i)^2}{\sum_{i \in I_R} h_i + \lambda} - \frac{(\sum_{i \in I} g_i)^2}{\sum_{i \in I} h_i + \lambda} \right] - \gamma
    \]
    \begin{itemize}
        \item Split if Gain $> 0$, choosing the feature and threshold that maximizes Gain.
    \end{itemize}
\end{itemize}

\textbf{Prediction}

\begin{itemize}
    \item Regression:
    \[
        \hat{y}_i = \sum_{t=1}^T f_t(x_i)
    \]

    \item Classification (Binary):
    \[
        \hat{y}_i = \sigma \left( \sum_{t=1}^T f_t(x_i) \right)
    \]
    where $\sigma(z) = \frac{1}{1+e^{-z}}$ is the sigmoid function for probability output.
\end{itemize}

\subsubsection{Advantages}
\begin{itemize}
    \item High Accuracy: Often outperforms other models on structured data due to sequential error correction.
    \item High Accuracy: Often outperforms other models on structured data due to sequential error correction.
    \item Handles Missing Data: Automatically learns how to treat missing values.
    \item Regularization: Prevents overfitting with L1/L2 penalties and tree pruning.
    \item Scalable: Optimized for speed with parallel processing and efficient tree construction.
    \item Feature Importance: Provides insights into which features drive predictions.
\end{itemize}

\subsubsection{Disadvantages}
\begin{itemize}
    \item Computationally Intensive: Training can be slow for large datasets with many trees.
    \item Complex Tuning: Requires careful tuning of hyperparameters (e.g., learning rate, max depth).
    \item Less Interpretable: Ensemble of trees is harder to interpret than a single tree.
    \item Poor with Sparse Data: Less effective for high-dimensional, sparse data (e.g., text) compared to Naive Bayes.
    \item Overfitting Risk: Without proper regularization, can overfit noisy data.
\end{itemize}

%    \section{Contribution}
%    \label{sec:contribution}
%
%    The following table represents the contribution of each member, note that whichever member handles whichever task will also write the report for that task.
%
%    \begin{table}[h]
%        \centering
%        \caption{Members Contributions}
%        \setlength{\tabcolsep}{2pt} % Reduce column spacing
%        \renewcommand{\arraystretch}{1} % Adjust row spacing
%        \resizebox{240}{!}{ % Fit within column width
%            \begin{tabular}{|l|c|c|c|}
%                \hline
%                \textbf{ID} & \textbf{Member} & \textbf{Contribution} & \textbf{Progress}\\
%                \hline
%                522H0036 & Luong Canh Phong & Task 1 and Handling Report & 100\%\\
%                522H0092 & Cao Nguyen Thai Thuan & Overseer and Report Support & 100\%\\
%                522H0075 & Tang Minh Thien An & Task 3 & 100\%\\
%                522H0167 & Truong Tri Phong & Task 2 & 100\%\\
%                \hline
%            \end{tabular}
%        }
%        \label{tab:contributions}
%    \end{table}
%
%    \section{Self-evaluation}
%    \label{sec:self-evaluation}
%
%    The following table is our self-evaluation on our tasks:
%
%    \begin{table}[h]
%        \centering
%        \caption{Self-evaluation}
%        \setlength{\tabcolsep}{2pt} % Reduce column spacing
%        \renewcommand{\arraystretch}{1} % Adjust row spacing
%        \resizebox{240}{!}{ % Fit within column width
%            \begin{tabular}{|l|c|c|c|}
%                \hline
%                \textbf{Task} & \textbf{Task Requirements} & \textbf{Completion Ratio}\\
%                \hline
%                Task 1 & A-Priori Algorithm for Frequent Customers & 100\%\\
%                Task 2 & PCY Algorithm for Frequent Items & 95\%\\
%                Task 3 & MinHashLSH for Similar Dates & 90\%\\
%                Task 4 & Report & 100\%\\
%                \hline
%            \end{tabular}
%        }
%        \label{tab:self-evaluation}
%    \end{table}
%
%    \section{Conclusion}
%    \label{sec:conclusion}
%    We have gone through a variety of techniques and algorithms used in the world of data mining.
%    For the first task, we have to find same-day customers and utilize the A-Priori algorithm to find frequent pairs of customers that shop on the same date and save the output of each pass in a dedicated folder.
%    As we run though the code, the result after sorting is a reasonable ascending list of frequent customer pairs.
%    For the second task, store the given dataset locally and identify baskets, as well as implementing the PCY algorithm to find frequent pairs along with generating metadata with predetermined constraints, the results for this task are two separate lists, one containing all frequent pairs, and the other is a list of association rules based on the user's given support threshold and confidence value.
%    And finally, implement and compare between a traditional and an alternative MinHashLSH function to understand and have a greater insight into how the frequent pairs searching is done.
%    We can see that with a slight modification and a different way of merging, it can result in a notably higher efficiency and better results.
%
%    \begin{thebibliography}{00}
%        \bibitem{b12} Tpoint Tech, ``Apriori Algorithm, ''\\\
%        [Online]. Available: \href{https://www.tpointtech.com/apriori-algorithm}{https://www.tpointtech.com/apriori-algorithm}
%        \bibitem{b6} Databricks, ``MapReduce, '' Databricks Glossary, 2025.\\\
%        [Online]. Available: \href{https://www.databricks.com/glossary/mapreduce}{https://www.databricks.com/glossary/mapreduce}
%        \bibitem{b1} J. S. Park and M. S. Chen, ``Using a hash table to eliminate candidates in a frequent itemset mining algorithm, '' \textit{IEEE Trans. Knowl. Data Eng.}, vol. 7, no. 3, pp. 464--472, 1995.
%        \bibitem{b2} J. Han, J. Pei, and Y. Yin, ``Mining frequent patterns without candidate generation, '' \textit{ACM SIGMOD Rec.}, vol. 29, no. 2, pp. 1--12, 2000.
%        \bibitem{b3} PySpark Documentation, ``PySpark API Documentation, '' 2025.\\\
%        [Online]. Available: \url{https://spark.apache.org/docs/latest/api/python}
%        \bibitem{b4} PySpark Documentation, ``pyspark.ml.feature.MinHashLSH, '' Apache Spark, 2025.\\\
%        [Online]. Available: \href{https://spark.apache.org/docs/latest/api/python/reference/api/pyspark.ml.feature.MinHashLSH}{https://spark.apache.org/docs/latest/api/python/refer\-ence/api/pyspark.ml.feature.MinHashLSH}
%        \bibitem{b5} Amazon Web Services, ``Jaccard similarity, '' AWS Neptune Analytics Documentation, 2024.\\\
%        [Online]. Available: \href{https://docs.aws.amazon.com/neptune-analytics/latest/userguide/jaccard-similarity.html}{https://docs.aws.amazon.com/neptune-analytics/latest/userguide/jaccard-similarity.html}
%    \end{thebibliography}
\end{document}