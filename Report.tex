%! Author = ACER
%! Date = 31-Mar-25

% Preamble
\documentclass[11pt]{article}

% Packages
\usepackage[utf8]{inputenc} % For special characters
\usepackage{amsmath}        % For mathematical formulas
\usepackage{amssymb}        % For mathematical symbols
\usepackage{graphicx}       % For including images
\usepackage{booktabs}       % For better tables
\usepackage{hyperref}       % For hyperlinks
\usepackage{geometry}       % For page layout
\usepackage{caption}        % For customizing figure captions
\usepackage{float}          % For better figure placement
\usepackage{listings}       % For code listings
\usepackage{xcolor}         % For colors
\usepackage{cite}           % For citation
\usepackage{lipsum}
\usepackage{algorithmicx}         % For dummy text

% Geometry
\geometry{a4paper, margin=1in} % Set page margins

% Title and Author Information
\title{Data Mining Report: [Your Report Title Here]}
\author{
    Luong Canh Phong\textsuperscript{1} \\
    522H0036
    \and
    Le Tan Huy\textsuperscript{2} \\
    522H0030
    \and
    Member\textsuperscript{3}\\
    MSSV
    \and
    Member\textsuperscript{4}\\
    MSSV
    \and
    Member\textsuperscript{5}\\
    MSSV
}
\date{\today} % Current date

% Affiliations
\newcommand{\affiliationOne}{Department of Computer Science, University of Example}
\newcommand{\affiliationTwo}{Department of Data Science, University of Example}
\newcommand{\affiliationThree}{Department of ..., University of Example}
\newcommand{\affiliationFour}{Department of ..., University of Example}
\newcommand{\affiliationFive}{Department of ..., University of Example}

% Customize caption
\captionsetup{font=small, labelfont=bf}

% Code Listing Style
\definecolor{codegreen}{rgb}{0,0.6,0}
\definecolor{codegray}{rgb}{0.5,0.5,0.5}
\definecolor{codepurple}{rgb}{0.58,0,0.82}
\definecolor{backcolour}{rgb}{0.95,0.95,0.92}

\lstdefinestyle{mystyle}{
    backgroundcolor=\color{backcolour},
    commentstyle=\color{codegreen},
    keywordstyle=\color{magenta},
    numberstyle=\tiny\color{codegray},
    stringstyle=\color{codepurple},
    basicstyle=\ttfamily\footnotesize,
    breakatwhitespace=false,
    breaklines=true,
    captionpos=b,
    keepspaces=true,
    numbers=left,
    numbersep=5pt,
    showspaces=false,
    showstringspaces=false,
    showtabs=false,
    tabsize=2
}

\lstset{style=mystyle}

% Document
\begin{document}

    \maketitle

    \begin{abstract}
        \lipsum[1] % Replace with your actual abstract
        This report presents the findings of a data mining project focused on [briefly describe the problem]. We employed [list the main techniques used] to analyze [describe the dataset]. The results demonstrate [summarize the key findings].
    \end{abstract}

    \footnotetext[1]{\affiliationOne}
    \footnotetext[2]{\affiliationTwo}
    \footnotetext[3]{\affiliationThree}
    \footnotetext[4]{\affiliationFour}
    \footnotetext[5]{\affiliationFive}

    \tableofcontents % Table of contents

    \section{Introduction}
    \label{sec:introduction}

    \subsection{Problem Statement}
    \label{subsec:problem}
    The proliferation of unsolicited bulk email, commonly known as spam, poses a significant challenge to email users and service providers. This project aims to develop an effective email spam classification system capable of accurately distinguishing between legitimate (ham) and spam emails, thereby improving user experience and email system efficiency. Option 2 (Slightly More Detailed): Email spam, characterized by unsolicited and often unwanted messages, represents a persistent problem for individuals and organizations. Spam not only clutters inboxes and wastes users' time but also poses security risks through phishing attempts and malware distribution. This project addresses the problem of email spam by developing a robust classification model that can automatically and accurately identify spam emails, allowing for their effective filtering and removal. Option 3 (More Comprehensive): The exponential growth of email communication has been accompanied by a corresponding increase in the volume of spam emails. These unsolicited messages, often containing advertisements, scams, or malicious content, degrade the user experience, consume valuable resources, and pose significant security threats. The challenge lies in developing a system that can reliably differentiate between legitimate emails (ham) and spam emails, despite the evolving tactics employed by spammers. This project seeks to address this challenge by designing, implementing, and evaluating a machine learning-based email spam classification system. The system will be trained on a labeled dataset of emails and will be evaluated based on its ability to accurately classify new, unseen emails as either spam or ham. Option 4 (Focus on Impact): Unsolicited bulk email, or spam, is a pervasive issue that negatively impacts email users and service providers. Spam wastes users' time, clutters inboxes, and can expose users to phishing scams and malware. For email service providers, spam consumes bandwidth and storage resources, and can damage their reputation. This project aims to mitigate the negative impacts of spam by developing a highly accurate email spam classification system. The system will be designed to automatically identify and filter spam emails, thereby improving the user experience, reducing security risks, and optimizing resource utilization for email service providers. Key Elements of a Good Problem Statement

    \subsection{Objectives}
    \label{subsec:objectives}
    \begin{itemize}
        \item Objective 1: [Describe the first objective]
        \item Objective 2: [Describe the second objective]
        \item ...
    \end{itemize}

    \section{Literature Review}
    \label{sec:literature}
    \lipsum[5-6] % Replace with your actual content
    \subsection{Related Work}
    \label{subsec:related}
    \lipsum[7-8] % Replace with your actual content

    \section{Methodology}
    \label{sec:methodology}

    \subsection{Data Description}
    \label{subsec:data}
    \lipsum[9] % Replace with your actual content
    \begin{itemize}
        \item Data Source: [Describe the source of your data]
        \item Data Size: [Describe the size of your dataset]
        \item Data Features: [List the key features in your dataset]
    \end{itemize}

    \subsection{Data Preprocessing}
    \label{subsec:preprocessing}
    \lipsum[10] % Replace with your actual content
    \begin{itemize}
        \item Cleaning: [Describe the cleaning steps]
        \item Transformation: [Describe the transformation steps]
        \item Reduction: [Describe the reduction steps]
    \end{itemize}

    \subsection{Data Mining Techniques}
    \label{subsec:techniques}
    \lipsum[11] % Replace with your actual content
    \subsubsection{Technique 1: [Name of Technique]}
    \label{subsubsec:technique1}
    \lipsum[12] % Replace with your actual content
    \begin{algorithm}
        \caption{Algorithm Name}
        \begin{algorithmic}[1]
            \State Input: [Input description]
            \State Output: [Output description]
            \State Initialize: [Initialization steps]
            \While{[Condition]}
                \State [Step 1]
                \State [Step 2]
            \EndWhile
            \State Return [Result]
        \end{algorithmic}
    \end{algorithm}

    \subsubsection{Technique 2: [Name of Technique]}
    \label{subsubsec:technique2}
    \lipsum[13] % Replace with your actual content

    \section{Results}
    \label{sec:results}

    \subsection{Evaluation Metrics}
    \label{subsec:metrics}
    \lipsum[14] % Replace with your actual content
    \begin{itemize}
        \item Metric 1: [Describe the first metric]
        \item Metric 2: [Describe the second metric]
        \item ...
    \end{itemize}

    \subsection{Results of Technique 1}
    \label{subsec:results1}
    \lipsum[15] % Replace with your actual content
    \begin{figure}[H]
        \centering
        \caption{Caption for Figure 1}
        \label{fig:figure1}
    \end{figure}

    \begin{table}[H]
        \centering
        \begin{tabular}{ccc}
            \toprule
            Column 1 & Column 2 & Column 3 \\
            \midrule
            Data 1   & Data 2   & Data 3   \\
            Data 4   & Data 5   & Data 6   \\
            \bottomrule
        \end{tabular}
        \caption{Caption for Table 1}
        \label{tab:table1}
    \end{table}

    \subsection{Results of Technique 2}
    \label{subsec:results2}
    \lipsum[16] % Replace with your actual content

    \section{Discussion}
    \label{sec:discussion}
    \lipsum[17-18] % Replace with your actual content

    \section{Conclusion}
    \label{sec:conclusion}
    \lipsum[19] % Replace with your actual content

    \section{Future Work}
    \label{sec:future}
    \lipsum[20] % Replace with your actual content

    \section*{References}
    \label{sec:references}
    \begin{thebibliography}{99}
        \bibitem{ref1} Author 1, Title 1, Journal 1, Year 1.
        \bibitem{ref2} Author 2, Title 2, Journal 2, Year 2.
        % Add more references here
    \end{thebibliography}

\end{document}