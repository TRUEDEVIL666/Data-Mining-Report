To understand why spam classification is important to our lives, we must first understand the spam emails and its impact on daily life and businesses.
\subsection{Different Types of Spam Emails}
\label{subsec:different-types-of-spam-emails}

There are various forms of spam, sent with different intentions and purposes.
But they are commonly grouped into:

\begin{itemize}
    \item Phishing Emails
    \item Email Spoofing
    \item Technical support scams
    \item Current event scams
    \item Marketing/advertising email
    \item Malware scam
\end{itemize}

\subsection{Problems with Spam Emails}
\label{subsec:problems-with-spam-emails}

According to statistics report in 2023, 160 billion spam emails are sent every day, which is 46\% of the 347 billion emails sent daily.
Out of which, the most common type being marketing/advertising emails, which take up around 36\%, followed up with promotional of adult content, around 31.7\% of total spam emails.
Despite scam and fraudulent emails being the least common type, over 70\% of them are phishing emails, which is still over 6 billion phishing emails are being sent to users daily.

A single spam email carbon emission is almost 0.03g of Carbon Dioxide Equivalent (CO$_{2}$e).
With the amount of spam being sent daily, it can easily get nearly 5 tonnes of CO$_{2}$e being released every day.
Additionally, two-thirds of spam receivers have been reported to have their mental health affected due to the number of spam or phishing scams.

For businesses, this spam can be sent as a way to get businesses to invest in nonexistent organizations under the guise of an investment and promise a high payback.
For individuals, it would be under the form of bitcoin investment or for a charitable cause.
Once the money is received, the sender would delete all traces and block the recipient contact.

\section{Methodologies for Spam Classification}
\label{sec:methodologies-for-spam-classification}

Many methods to prevent spam are applied; a commonly used method is using Machine Learning models like Random Forest (RF), Support Vector Machine (SVM), Logistic Regression (LR), Naive Bayes (NB) or Deep Learning models such as Artificial Neural Network (ANN),
(Explantion of used algorithms)