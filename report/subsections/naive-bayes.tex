Naive Bayes is a family of simple, probabilistic, supervised machine learning algorithms used primarily for classification tasks, though it can be adapted for other purposes.
It is based on Bayes’ Theorem and assumes that features are conditionally independent given the class label (the ``naive'' assumption).
Despite this simplifying assumption, Naive Bayes performs surprisingly well in many real-world applications, especially in text classification and spam filtering.

The core idea is: Given a set of features, predict the most likely class by calculating probabilities based on prior observations.

\subsubsection{Basic Components}

\begin{itemize}
    \item \textbf{Training Data}:
        \[D = \{(x_1, y_1), (x_2, y_2), \dots, (x_N, y_N)\}\]
        where $x_i = (x_{i1}, x_{i2}, \dots, x_{id}) \in \mathbb{R}^d$ is a feature vector with d dimensions, and $y_i \in \{C_1, C_2, \dots, C_k\}$ is a class label from k classes.
    \item \textbf{Bayes' Theorem}: The foundation for calculating probabilities of classes given features.
    \item \textbf{Conditional Independence Assumption}: Assumes that features $x_{i1}, x_{i2}, \dots, x_{id}$ are independent given the class $y$.
    \item \textbf{Prior Probability}: The probability of each class before observing the features.
    \item \textbf{Likelihood}: The probability of observing the features given a class.
    \item \textbf{Posterior Probability}: The probability of a class given the observed features, which is used for prediction.
    \item \textbf{Smoothing}: A technique (e.g., Laplace smoothing) to handle zero probabilities for unseen feature-class combinations.
\end{itemize}

\subsubsection{Detailed Algorithm}

Here are the steps to implement and use the Naive Bayes algorithm:

\begin{enumerate}[label=Step \arabic*:, align=left, leftmargin=20pt,labelsep=1em]
    \item Data Preparation
    \begin{itemize}
        \item Prepare the dataset $D$, where each sample has features $x_i$ and a class label $y_i$.
        \item For categorical features, Naive Bayes is straightforward.
        For continuous features, assume a distribution (e.g., Gaussian) or discretize the data.
    \end{itemize}

    \item Training (Model Building)
    \begin{itemize}
        \item Estimate Prior Probabilities: Calculate the probability of each class $C_j$:
        \[P(C_j) = \frac{\text{Number of samples with class } C_j}{\text{Total number of samples}}\]

        \item Estimate Likelihoods: For each feature $x_m$ and class $C_j$, compute the conditional probability $P(x_m | C_j)$:
        \begin{itemize}
            \item Categorical Features: Use frequency counts.
            \item Continuous Features: Assume a distribution (e.g., Gaussian) and estimate parameters (mean, variance).
            \item Apply smoothing (e.g., Laplace smoothing) to avoid zero probabilities.
        \end{itemize}

        \item Store these probabilities for use during prediction.
    \end{itemize}

    \item Prediction
    \begin{itemize}
        \item For a new input $x = (x_1, x_2, \dots, x_d)$:
        \begin{enumerate}
            \item Compute the posterior probability for each class $C_j$ using Bayes' Theorem:
                \[P(C_j | x) \propto P(C_j) \prod_{m=1}^d P(x_m | C_j)\]
                (The proportionality $\propto$ is used because the denominator P(x) is constant across classes.)

            \item Predict the class with the highest posterior probability:
                \[\hat{y} = \arg\max_{C_j} P(C_j) \prod_{m=1}^d P(x_m | C_j)\]

            \item Optionally, compute normalized probabilities by dividing by the sum of all class posteriors
        \end{enumerate}
    \end{itemize}
\end{enumerate}


\subsubsection{Mathematical Formulas}\text{}

\smallskip
Naive Bayes is grounded in Bayes' Theorem and the conditional independence assumption.
Below are the key formulas, explained intuitively.

\textbf{Bayes' Theorem}

For a class $C_j$ and feature vector $x = (x_1, x_2, \dots, x_d)$:
\[P(C_j | x) = \frac{P(C_j) P(x | C_j)}{P(x)}\]

\begin{itemize}[leftmargin=*] % Start list explaining terms
    \item $P(C_j | x)$: Posterior probability (probability of class $C_j$ given features $x$).
    \item $P(C_j)$: Prior probability (probability of class $C_j$ before seeing $x$).
    \item $P(x | C_j)$: Likelihood (probability of observing $x$ given class $C_j$).
    \item $P(x)$: Evidence (probability of observing $x$, a normalizing constant).
\end{itemize}

Since $P(x)$ is the same for all classes, we can ignore it for classification:
\[P(C_j | x) \propto P(C_j) P(x | C_j)\]

\textbf{Conditional Independence Assumption}

Naive Bayes assumes that features are independent given the class:
\[P(x | C_j) = P(x_1, x_2, \ldots, x_d | C_j) = \prod_{m=1}^d P(x_m | C_j)\]

\textbf{Intuition:} If you know the class (e.g., ``spam email''), the presence of one feature (e.g., the word ``free'') doesn't affect the probability of another feature (e.g., the word ``win'').
This is often unrealistic but simplifies calculations.

\smallskip
\textbf{Prior Probability}
\[P(C_j) = \frac{\text{Count}(y = C_j)}{N}\]
where $\text{Count}(y = C_j)$ is the number of samples with class $C_j$, and $N$ is the total number of samples.

\smallskip
\textbf{Likelihood}

\begin{itemize}
    \item Categorical Features:
        \[P(x_m = v | C_j) = \frac{\text{Count}(x_m = v, y = C_j)}{\text{Count}(y = C_j)}\]
        where $v$ is a specific value of feature $x_m$.

    \item Laplace Smoothing: To avoid zero probabilities:
        \[P(x_m = v | C_j) = \frac{\text{Count}(x_m = v, y = C_j) + \alpha}{\text{Count}(y = C_j) + \alpha \cdot |\text{Values}(x_m)|}\]
        where $\alpha$ (e.g., 1) is the smoothing parameter, and $|\text{Values}(x_m)|$ is the number of possible values for $x_m$.

    \item Continuous Features (Gaussian Naive Bayes): Assume feature $x_m$ follows a Gaussian distribution for class $C_j$:
          \[P(x_m | C_j) = \frac{1}{\sqrt{2\pi \sigma^2_{mj}}} \exp\left( -\frac{(x_m - \mu_{mj})^2}{2\sigma^2_{mj}} \right)\]
          where:
    \begin{itemize}
        \item $\mu_{mj}$: Mean of feature $x_m$ for class $C_j$.
        \item $\sigma^2_{mj}$: Variance of feature $x_m$ for class $C_j$.
    \end{itemize}
\end{itemize}

\smallskip
\textbf{Error Metrics}
\begin{itemize}
    \item Classification Error:
        \[\text{Error} = \frac{1}{N_{\text{test}}} \sum_{i=1}^{N_{\text{test}}} I(\hat{y}_i \neq y_i)\]

    \item Log Loss (if probabilities are used):
        \[\text{Log Loss} = -\frac{1}{N_{\text{test}}} \sum_{i=1}^{N_{\text{test}}} \sum_{j=1}^{k} I(y_i = C_j) \log P(C_j | x_i)\]
\end{itemize}


\subsubsection{Advantages}

\smallskip
\begin{itemize}
    \item Simple and Fast: Easy to implement and computationally efficient, especially for training.
    \item Effective with Small Datasets: Performs well even with limited data, unlike complex models.
    \item Handles High-Dimensional Data: Common in text classification (e.g., bag-of-words models).
    \item Probabilistic Output: Provides probability estimates for each class, useful for decision-making.
    \item Robust to Irrelevant Features: The independence assumption mitigates the impact of irrelevant features.
\end{itemize}

\subsubsection{Disadvantages}
\begin{itemize}
    \item Naive Assumption: The conditional independence assumption is often unrealistic, leading to suboptimal performance when features are correlated.
    \item Sensitive to Zero Probabilities: Requires smoothing to handle unseen feature-class combinations.
    \item Poor with Continuous Features: Gaussian assumptions may not fit all data distributions.
    \item Outperformed by Complex Models: Often less accurate than SVM or Random Forest on complex datasets.
    \item Imbalanced Data Issues: May favor majority classes without proper adjustments.
\end{itemize}
