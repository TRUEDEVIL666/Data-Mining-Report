Random Forest is a popular machine learning model in the Ensemble Learning family, used for both classification and regression tasks.
It consists of a collection of decision trees trained on different subsets of data, with the final prediction aggregated from these trees.
The core idea is to combine multiple weak learners (individual decision trees) into a strong learner with higher accuracy and reduced overfitting compared to a single decision tree.

Random Forest relies on two key principles:

\begin{itemize}
    \item \textbf{Bagging (Bootstrap Aggregating)}: Creates multiple data subsets by randomly sampling with replacement from the original dataset.
    \item \textbf{Feature Randomness}: At each split in a decision tree, only a random subset of features is considered.
\end{itemize}

\subsubsection{Basic Components}\text{}

\smallskip
\textbf{Decision Tree}: The fundamental unit of Random Forest.
Each tree is built independently on a data subset.

\textbf{Forest}: A collection of many decision trees (typically hundreds or thousands).

\textbf{Bootstrap Sample}: Each tree is trained on a randomly sampled subset of the original dataset.

\textbf{Random Feature Subset}: At each node, only a random subset of features is evaluated for the best split.

\smallskip
\subsubsection{Detailed Algorithm}\text{}

\begin{enumerate}[label=Step \arabic*:, align=left, leftmargin=20pt,labelsep=1em]
    \item Data Preparation
    \begin{itemize}
        \item Assume an original dataset $D$ with $N$ samples and $M$ features.
    \end{itemize}

    \item Create Bootstrap Samples
    \begin{itemize}
        \item Generate $T$ subsets $D1, D2, \ldots,DT$ (where $T$ is the number of trees), each of size $N$, by randomly sampling with replacement from $D$.
        \item Due to sampling with replacement, some samples may appear multiple times, while others may not appear (approximately 36.8\% of the data is left out, called Out-of-Bag (OOB) data).
    \end{itemize}

    \item Build Each Decision Tree
    \begin{itemize}
        \item For each subset $D_t$:
        \begin{enumerate}
            \item Initialize a decision tree: Trees are grown fully without pruning.
            \item At each node:
            \begin{itemize}
                \item Randomly select a subset of $m$ features from the total $M$ features ($m \textless M$). Typically:
                \begin{itemize}
                    \item For classification: $m = \sqrt{M}$
                    \item For regression: $m = M/3$
                \end{itemize}
                \item Find the best feature among these $m$ features to split the node, based on criteria like Gini Index, Entropy (classification), or Mean Squared Error (regression).
            \end{itemize}
            \item Repeat the splitting process until the tree is complete (reaches maximum depth or cannot split further).
        \end{enumerate}
        \item Result: $T$ independent decision trees.
    \end{itemize}

    \item Aggregate Results

    \smallskip
    For classification:
    \begin{itemize}
        \item Each tree predicts a class.
        \item The final class is the one with the most votes (Majority Voting).
    \end{itemize}

    \smallskip
    For regression:
    \begin{itemize}
        \item Each tree predicts a numerical value.
        \item The final value is the average of all predictions.
    \end{itemize}

    \item Model Evaluation (Using OOB Error)
    \begin{itemize}
        \item OOB data (not used to train a given tree) is used to test the performance of each tree.
        \item Aggregate OOB results to estimate the overall accuracy of the Random Forest without a separate test set.
    \end{itemize}
\end{enumerate}

\smallskip
\subsubsection{Mathematical Formulas}\text{}

There are splitting criteria at each node in the tree, and it is different for each task.

\textbf{For Classification:}
\begin{itemize}
    \item Gini Index:
        \[Gini = 1 - \sum_{i=1}^C{p_i^2}\]
        where $p_i$ is the proportion of class $i$ in the node.
    \item Entropy:
        \[Entropy = - \sum_{i=1}^C{p_i \log_2{p_i}}\]
    \item Final Result:
        \[\hat{y} = mode(\hat{y}_1, \hat{y}_2, \ldots, \hat{y}_t)\]
        where $\hat{y}_t$ is the prediction from a tree $t$.
\end{itemize}

\textbf{For Regression:}
\begin{itemize}
    \item Splitting criterion: Minimize Mean Squared Error:
        \[MSE = \frac{1}{n} \sum_{i=1}^n{(y_i - \hat{y}_i)^2}\]
    \item Final Result:
        \[\hat{y} = \frac{1}{T} \sum_{t=1}^T{\hat{y}_t}\]
\end{itemize}

\smallskip
\subsubsection{Advantages}

\begin{itemize}
    \item High accuracy due to combining multiple trees.
    \item Resistance to overfitting thanks to randomness and aggregation.
    \item Handles large datasets with many features and samples effectively.
    \item No need for data normalization since it is based on decision trees.
    \item Provides feature importance measurement based on impurity reduction.
\end{itemize}

\smallskip
\subsubsection{Disadvantages}

\begin{itemize}
    \item Resource-intensive: Requires significant memory and computation for large numbers of trees.
    \item Less interpretable than a single decision tree.
    \item Reduced performance with linear data, where linear regression might be more suitable.
\end{itemize}