SVM is a supervised machine learning algorithm used for classification and regression, though it is primarily applied to classification tasks.
The core idea of SVM is to find the optimal hyperplane that best separates data points of different classes in a high-dimensional space, maximizing the margin between the classes.
For cases where data is not linearly separable, SVM uses the kernel trick to transform the data into a higher-dimensional space where a linear boundary can be established.

SVM is known for its robustness and effectiveness in handling high-dimensional datasets and is widely used in tasks like text classification, image recognition, and bioinformatics.

\subsubsection{Basic Components}\textbf{}

\smallskip
\textbf{Hyperplane}: A decision boundary that separates data points of different classes.
In d-dimensional space, a hyperplane is defined by $w^t +b = 0$, where w is the weight vector, b is the bias, and x is the input vector.

\textbf{Margin}: The distance between the hyperplane and the nearest data point from either class.
SVM aims to maximize this margin.

\textbf{Support Vectors}: The data points closest to the hyperplane, which define the margin and are critical to determining the hyperplane’s position.

\textbf{Kernel Function}: A function that transforms non-linearly separable data into a higher-dimensional space where a linear boundary can be found (e.g., linear, polynomial, or radial basis function (RBF) kernels).

\textbf{Regularization Parameter $(C)$}: Controls the trade-off between maximizing the margin and minimizing classification errors.

\textbf{Slack Variables}: Allow for some misclassification in soft-margin SVM to handle non-linearly separable data.

\smallskip
\subsubsection{Detailed Algorithm}\textbf{}

\begin{enumerate}[label=Step \arabic*:, align=left, leftmargin=20pt,labelsep=1em]
    \item Data Preparation
    \begin{itemize}
        \item Assume an original dataset
            \[D = \{ (x_1, y_1), (x_2, y_2), \dots, (x_N, y_N) \}\]
            where $x_i \in \mathbb{R}^d$ (feature vector with $d$ dimensions) and $y_i \in \{-1, +1\}$ (binary class labels).
    \end{itemize}

    \item Define the Objective

    \smallskip
    Hard-Margin SVM (Linearly Separable Data):
    \begin{itemize}
        \item Find the hyperplane $w^T x + b = 0$ that separates the classes with the maximum margin.
        \item The margin is defined as the distance from the hyperplane to the nearest data point, given by $\frac{2}{\|w\|}$, where $\|w\| = \sqrt{w^T w}$.
        \item Objective: Maximize the margin, i.e., minimize $\frac{1}{2} \|w\|^2$, subject to:
            \[y_i (w^T x_i + b) \geq 1, \quad \forall i = 1, \dots, N\]
    \end{itemize}

    \smallskip
    Soft-Margin SVM (Non-Linearly Separable Data):
    \begin{itemize}
        \item Introduce slack variables $\xi_i \geq 0$ to allow some misclassification.
        \item Objective: Minimize:
            \[\frac{1}{2} \|w\|^2 + C \sum_{i=1}^N \xi_i\]
            Subject to:
            \[y_i (w^T x_i + b) \geq 1 - \xi_i, \quad \xi_i \geq 0, \quad \forall i\]
            where $C$ is the regularization parameter controlling the trade-off between margin maximization and classification error.
    \end{itemize}

    \item Solve the Optimization Problem

    \smallskip
    \begin{itemize}
        The optimization problem is typically solved in its \textbf{dual form} using Lagrange multipliers to handle constraints efficiently.
        \item Dual Problem:
        \begin{itemize}
            \item Introduce Lagrange multipliers $\alpha_i \geq 0$.
            \item The dual optimization problem is:
        \end{itemize}

        Maximize
        \[L(\alpha) = \sum_{i=1}^N \alpha_i - \frac{1}{2} \sum_{i=1}^N \sum_{j=1}^N \alpha_i \alpha_j y_i y_j (x_i^T x_j)\]

        Subject to:
        \[\sum_{i=1}^N \alpha_i y_i = 0, \quad 0 \leq \alpha_i \leq C, \quad \forall i\]

        \item The solution $\alpha_i$ determines the weight vector:
            \[w = \sum_{i=1}^N \alpha_i y_i x_i\]
        \item The bias $b$ is computed using support vectors (where $\alpha_i > 0$).
        \item Support vectors are the points where $y_i(w^T x_i + b) = 1$ (on the margin) or $\xi_i > 0$ (misclassified or within the margin).
    \end{itemize}

    \item Handle Non-Linear Data with Kernels
    \begin{itemize}
        For non-linearly separable data, map the input data to a higher-dimensional space using a kernel function $K(x_i, x_j)$.
        \item Common kernels:
        \begin{itemize}
            \item Linear Kernel:
                \[K(x_i, x_j) = x_i^T x_j\]
            \item Polynomial Kernel:
                \[K(x_i, x_j) = (x_i^T x_j + c)^d\]
            \item Radial Basis Function (RBF) Kernel:
                \[K(x_i, x_j) = \exp(-\gamma \|x_i - x_j\|^2)\]
        \end{itemize}

        \item The dual problem becomes: Maximize
            \[L(\alpha) = \sum_{i=1}^N \alpha_i - \frac{1}{2} \sum_{i=1}^N \sum_{j=1}^N \alpha_i \alpha_j y_i y_j K(x_i, x_j)\]
            Subject to the same constraints as above.
    \end{itemize}

    \item Prediction

    For a new input $x$:
    \begin{itemize}
        \item Compute the decision function:
            \[f(x) = \sum_{i \in \text{SV}} \alpha_i y_i K(x_i, x) + b\]
            where SV is the set of support vectors.
        \smallskip
        \item Predict the class:
            \[\hat{y} = sign(f(x))\]
        \item For probability estimates (if enabled), use techniques like Platt scaling to convert f(x) into probabilities
    \end{itemize}
\end{enumerate}

\smallskip
\subsubsection{Advantages}\textbf{}

\begin{itemize}
    \item Effective in High-Dimensional Spaces: Works well with datasets having many features (e.g., text classification).
    \item Robust to Outliers: Maximizing the margin focuses on support vectors, ignoring points far from the boundary.
    \item Flexible with Kernels: The kernel trick allows SVM to handle non-linearly separable data effectively.
    \item Global Optimization: The convex optimization problem ensures a unique solution.
    \item Sparse Solution: Only support vectors (a subset of the data) determine the model, making it memory-efficient.
\end{itemize}

\subsubsection{Disadvantages}\textbf{}

\begin{itemize}
    \item Computationally Expensive: Training time scales poorly with large datasets ($O(N^2) to O(N^3)$ for solving the quadratic optimization problem).
    \item Sensitive to Parameter Tuning: Requires careful tuning of $C$ and kernel parameters (e.g., γ for RBF kernel).
    \item Not Interpretable: The resulting hyperplane and support vectors are not as intuitive as decision trees.
    \item Poor with Noisy Data: Overlapping classes or noisy data can degrade performance.
    \item Not Ideal for Large Datasets: Due to high computational cost, SVM is less suitable for massive datasets compared to models like Random Fores
\end{itemize}